\documentclass[a4paper, 12pt]{article}

\usepackage{amsmath}
\usepackage[margin=20mm]{geometry}
\usepackage[spanish]{babel}
%\selectlanguage{spanish}
%\usepackage[utf8]{inputenc}
\usepackage{fancyhdr}
\usepackage{anysize}
\renewcommand\refname{Bibliograf\'ia}
\marginsize{1.5cm}{1.5cm}{1cm}{4cm}

\begin{document}

Gravedad cuática 2+1 dimensiones a 3 loops.


\section*{Capítulo 1: Introduccion}

En tres dimensiones el tensor de curvatura queda determinado por el tensor de Ricci:

\begin{align*}
&R_{\mu\nu}^{\rho\sigma} = \delta_\mu^\rho R_\nu^\sigma + \delta_\nu^\sigma R_\mu^\rho - \delta_\nu^\rho R_\mu^\sigma - \delta_\mu^\sigma R_\nu^\rho - \frac{1}{2}(\delta_\mu^\rho \delta_\nu^\sigma - \delta_\mu^\sigma \delta_\nu^\rho) R\\
&R_{\mu\nu} = 2\Lambda g_{\mu\nu}
\end{align*}

En tanto que el tensor de Weyl se anula completamente.

PREGUNTAR ESTO:\

==========

Si se satisfacen las ecuaciones de Einstein (se refiere a dadas cond. de contorno?), la curvatura queda determinada por el tensor de energía-momento.

Explicacion de grados de libertad locales.

Se pueden fijar 3 por eleccion de coordenadas. ¿Se refiere a diagonalizar?

¿Por que $G_\mu^0 = k^2 T_\mu^0$ son constraints?

Diferencia en grados de libertad fisicos y gravitacionales.

==========

La teoría no tiene limite newtoniano.
Modelo simple pero de interes dado la existencia de grados de libertad gravitacionales.

\begin{enumerate}
    \item Grados de libertad geómetricos en espacio-tiempo no trivial.
    \item Grados de libertad de gauge.
\end{enumerate}



\section*{Capítulo 2: Gravedad clásica y cuántica en 2+1}

Con constante cosmológica, sin materia.

\begin{align*}
&I_{grav} = \frac{1}{2\kappa^2} \int _M d^3 \sqrt{|g|}(R - 2\Lambda)
\end{align*}

==========

VER ESTRUCTURAS GEOMETRICAS "DESDE CERO"

==========


Formalismo de Chern-Simmons

Formalismo de primer orden.

¿A que refiere con variables fundamentales? ¿Spin connection?

\begin{align*}
&e^a = e^a_\mu dx^\mu\\
&\omega^a = \frac{1}{2}\epsilon^{abc} \omega_{\mu bc} dx^\mu
\end{align*}

Siendo $g_{\mu\nu} = \eta_{ab} e^e_{\ \mu} e^b_{\ \nu}$

La accion de Einstein-Hilbert queda:

\begin{align*}
&I = \frac{1}{\kappa^2} \int_M {e^a \wedge (d\omega_a + \frac{1}{2} \epsilon_{abc} \omega^b \wedge \omega^c) - \frac{\Lambda}{6} \epsilon_{abc} e^a \wedge e^b \wedge e^c}
\end{align*}


\vspace{1cm}

Formalismo ADM

\vspace{1cm}

Caso espacialmente abierto

\vspace{1cm}

\subsection*{Gravedad cuántica}

\vspace{1cm}

\section*{Capítulo 3: Teoría de perturbaciones en gravedad en 2+1}

\vspace{1cm}

\section*{Capítulo 4:}

Cuenta y diagramas a 2 y 3 loops.
Regularización dimensional.
$Temperatura \neq 0$
Caso D dimensional.

\vspace{1cm}

\section*{Capítulo 5:}

Sintesis y preguntas abiertas.


Referencias:

\end{document}
